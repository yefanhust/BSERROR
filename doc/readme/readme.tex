\documentclass{article}

\usepackage[top=1in,bottom=1in,left=1in,right=1in]{geometry}
\usepackage{graphicx}
\usepackage{booktabs}
\usepackage{amsmath}
\usepackage{amssymb}
\usepackage[utf8]{inputenc}

\begin{document}

\title{\textbf{Read Me}}
\author{YE Fan}

\maketitle

The included \emph{\textbf{makefile}} is designed for \emph{\textbf{Linux}}. 

In order to compile the code, run\footnote{Please note that it is required to have Intel compiler installed in the system.}

\begin{center}
  \emph{make all}
\end{center}

In order to run all benchmarks, run\footnote{Also please note that the presence of Intel Xeon Phi coprocessor and a running MIC Platform Software Stack (MPSS) is indispensable to run *.mic}

\begin{center}
  \emph{make run}
\end{center}

\section*{File list}
The \emph{BSERROR} directory contains all the files intended for PAC'14. The history of development is kept using \emph{GIT}. Here's a brief description for each file.

\begin{description}
\item[BSERROR] \hfill\\
  .git: git repository\\
  .gitignore: instruct git to ignore untracked files\\
  makefile: makefile for the programs\\
  run.sh: a script called by makefile, please not to run it directly\\
  seq.c: serial version of the model\\
  par\_omp.c: 1st OpenMP version\\
  par\_omp2.c: 2nd OpenMP version\\
  par\_mpi.c: Hybrid MPI/OpenMP version, which is the final goal of this project\\
  
\item[BSERROR/inc] header files\\
  misc.h: implementations of some auxiliary routines\\
  onetimesimu.h: implementation of one-time Monte Carlo simulation taking the par\_omp2.c approach, along with some relative routines\\
\item[BSERROR/doc] documents\\
  readme: tex files for generating README.pdf\\
  report: tex files for generating report.pdf\\
\item[BSERROR/ParameterAnalysis] various experimental results and figures\\
\end{description}


\section*{Mathematical description of the project}
Equation~\ref{onetimesimu} gives the discrete time hedging error for one time simulation. The goal is to simulate M times where M is 
sufficiently large so that the $Prob$ tends to be stable. The $Prob$ is defined as the number of times out of M when error is less than 
an accepted value $\epsilon$. 
\begin{equation}
\label{onetimesimu}
\begin{split}
M_T^N
&=e^{-rT}f(X_T)-(u(0,x)+\int_0^T\frac{\partial u}{\partial x}(\varphi(t), 
X_{\varphi_t}))d\widetilde{X}_t\\
&=\int_0^T\frac{\partial u}{\partial x}(t, X_t)d\widetilde{X}_t-\int_0^T\frac{\partial u}{\partial x}(\varphi(t), X_{\varphi(t)})d\widetilde{X}_t\\
&=(X_T-K)^+-\mathbb{E}[(X_T-K)^+]-\int_0^T\frac{\partial u}{\partial x}(\varphi(t), X_{\varphi(t)})dX_t\\
&=(X_T-K)^+-x_0N(d_1(0))+KN(d_2(0))-\sum_{i=0}^{n-1}N(d_1(t_i))(X_{t_{i+1}}-X_{t_i})\\
&=(X_T-K)^+-\frac{x_0}{\sqrt{2\pi}}\int_{-\infty}^{\frac{\log(\frac{x_0}{K})+\frac{1}{2}\sigma^2T}
{\sigma\sqrt{T}}}e^{-\frac{v^2}{2}}dv+\frac{K}{\sqrt{2\pi}}\int_{-\infty}^{\frac{\log(\frac{x_0}{K})-\frac{1}{2}\sigma^2T}{\sigma\sqrt{T}}}e^{-\frac{v^2}{2}}dv\\
&-\sum_{i=0}^{n-1}\frac{1}{\sqrt{2\pi}}\int_{-\infty}^{\frac{\log(\frac{X_{t_i}}{K})+\frac{1}{2}\sigma^2(T-t_i)}{\sigma\sqrt{T-t_i}}}e^{-\frac{v^2}{2}}dv(X_{t_{i+1}}-X_{t_i})\\
\end{split}
\end{equation}

The $X_{t}$ is the price of the underlying asset for the option, which is assumed to evolve in time according to the stochastic equation
\begin{equation}
  dX(t) = \mu X(t)dt + \sigma X(t)dB(t)
\end{equation}

In this equation, $\mu$ is the drift of the asset, $\sigma$ is the option volatility, and $B(t)$ is a standard Brownian motion.

The solution of this stochastic differential equation can be written as 
\begin{equation}
  X_{t_i}=X_{t_{i-1}}e^{(\mu-\sigma^2/2)\delta t+\sigma\sqrt{\delta t}\chi}
\end{equation}
where $\chi$ is a normally distributed random variable with zero mean and unit standard deviation, and \\$\delta t = t_i - t_{i-1}$.



By our unpublished theory, the upper bound of $N$ (discrete intervals of hedging) is given by Equation~\ref{upperbound}, which's the best estimation (the lowest upper bound) for the time being. 
\begin{equation}
\label{upperbound}
  N_{max} = \log^3{(1-Prob)e^{\frac{1}{4}}\frac{1}{T^{\frac{1}{4}}\sqrt{\frac{\epsilon}{(\log{\frac{X_0}{K}}+0.5\sigma^2T)\sqrt{2\pi}}}}}\cdot(-\frac{8e^3X_0^2\cdot 16 \cdot e^{\sigma^2}}{27\epsilon^2\pi})
\end{equation}

There's one more thing to mention, we've implemented our own gaussian integral function using Simpson's rule. See equations~\ref{simpson} and \ref{gauss}.

\begin{equation}
\label{simpson}
\int_{a}^{b}f(x)dx\approx \frac{h}{3}\left[f(x_0)+2\sum\limits_{j=1}^{n/2-1}f(x_{2j})+4\sum\limits_{j=1}^{n/2}f(x_{2j-1})+f(x_n)\right]
\end{equation}

\begin{equation}
\label{gauss}
f(x)=\int_{-\infty}^{x}e^{-\frac{t^2}{2}}dt=0.5\times \sqrt{2\pi} + \int_{0}^{x}e^{-\frac{t^2}{2}}dt
\end{equation}


\end{document}
